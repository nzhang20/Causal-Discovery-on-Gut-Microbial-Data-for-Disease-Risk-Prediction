% Gemini theme
% https://github.com/anishathalye/gemini
% !TEX TS-program = lualatex

\documentclass[final]{beamer}

% ====================
% Packages
% ====================

\usepackage[T1]{fontenc}
\usepackage{lmodern}
\usepackage[size=custom,width=48,height=36,scale=0.4]{beamerposter}
\usetheme{gemini}
\usecolortheme{gemini}
\usepackage{graphicx}
\usepackage{booktabs}
\usepackage{tikz}
\usepackage{pgfplots}
\pgfplotsset{compat=1.14}
\usepackage{anyfontsize}

% ====================
% Lengths
% ====================

% If you have N columns, choose \sepwidth and \colwidth such that
% (N+1)*\sepwidth + N*\colwidth = \paperwidth
\newlength{\sepwidth}
\newlength{\colwidth}
\setlength{\sepwidth}{0.025\paperwidth}
\setlength{\colwidth}{0.3\paperwidth}

\newcommand{\separatorcolumn}{\begin{column}{\sepwidth}\end{column}}

% ====================
% Title
% ====================

\title{\centering Causal Discovery on Gut Microbial Data for Disease Risk Prediction}

\author{Mariana Paco Mendivil \inst{1} \and Candus Shi \inst{2} \and Nicole Zhang \inst{3} \and Mentor: Dr. Biwei Huang \inst{4} \and Mentor: Dr. Jelena Bradic \inst{5}}
 
\institute[shortinst]{\inst{1} mpacomendivil@ucsd.edu \samelineand \inst{2} c6shi@ucsd.edu \samelineand \inst{3} nwzhang@ucsd.edu \samelineand \inst{4} bih007@ucsd.edu \samelineand \inst{5} jbradic@ucsd.edu}

% \author{\vspace{1cm} \hspace{1cm} \raggedright Mariana Paco Mendivil \and Candus Shi \and Nicole Zhang }

% \institute {\hspace{1.6cm} \raggedright mpacomendivil@ucsd.edu \samelineand \hspace{1.2cm} c6shi@ucsd.edu \hspace{1cm} nwzhang@ucsd.edu}

% ====================
% Footer (optional)
% ====================

\footercontent{
  Please visit our website at \href{https://nzhang20.github.io/Causal-Discovery-on-Gut-Microbial-Data-for-Disease-Risk-Prediction/}{https://nzhang20.github.io/Causal-Discovery-on-Gut-Microbial-Data-for-Disease-Risk-Prediction/}}
% (can be left out to remove footer)

% ====================
% Logo (optional)
% ====================

% use this to include logos on the left and/or right side of the header:
\logoright{\includegraphics[height=1.5cm]{hdsi-white.png}}
% \logoleft{\includegraphics[height=7cm]{logo2.pdf}}

% ====================
% Body
% ====================

\begin{document}

\begin{frame}[t]
\begin{columns}[t]
\separatorcolumn

\begin{column}{\colwidth}

  \begin{block}{Background}
    \begin{itemize}
      \item \textbf{Association vs. Causation}
      \text{}
      \item \textbf{Causal Discovery Algorithms in Gut Microbiome Studies}
      Previous research has explored causal discovery in gut microbiome studies, notably using algorithms like PC-stable to construct causal networks \\
      and implement do-calculus for estimating microbe-microbe and microbe-outcome causal effects. More recent advancements include CD-NOD, \\
      specifically designed for heterogeneous data, which is particularly valuable for gut microbiome research where samples often come from different studies. \\
      These algorithms aim to identify true causal relationships by performing conditional independence tests and orienting edges using established rules, while accounting \\
      for the unique challenges of microbiome data analysis.
    \end{itemize}

  \end{block}

  \begin{block}{Research Questions}

    \begin{enumerate}
      \item \textbf{Microbe-Microbe:} How do the microbe-microbe interaction networks between the healthy and diseased participants differ?
      \item \textbf{Microbe-Disease:} What microbes have a causal relationship to disease status?
      \item \textbf{Prediction:} Is it possible to predict disease status with the current composition of the dataset given causal representation learning techniques? \\
      How do they differ with the microbes learned in question 2?
    \end{enumerate}
    %Eget augue porta, bibendum venenatis tortor.
  \end{block}

  \begin{block}{Data}

    To answer the questions above, we apply our framework to gut microbial data that investigated T2D and 
    an individual participant data meta-analysis dataset that investigated PCOS. 
    
    \begin{itemize}
      \item \textbf{T2D:} For T2D, we use the NIH Human Microbiome Project (HMP2) dataset, filtered on healthy samples and individuals with a known insulin resistance or insulin sensitivity status.
      \item \textbf{PCOS:} For PCOS, we use a dataset aggregated from 14 different clinical studies. (elaborate)
    \end{itemize}

  \end{block}
  
  \begin{alertblock}{Causal Discovery}
  
  	Causal discovery is all about recovering the true causal structure of system given observed data. One way to model this causal structure is through a graph One of the oldest and most widely-used general-purpose causal discovery algorithms
	is PC. 
  
  \end{alertblock}

\end{column}

\separatorcolumn

\begin{column}{\colwidth}

  \begin{block}{Methods}

   In this study, we use causal discovery algorithms and compare them with 
   predictive modeling to explore the causal relationships between the 
   gut microbiome and two diseases: T2D and PCOS. Due to the high-dimensionality
   of the data and small sample sizes, we first select features through sparse
   estimation methods and sure-screening to reduce the number of microbes. 

    \begin{enumerate}
      \item \textbf{Filter out rare OTUs}. Remove microbes where all samples have less than 1\%
      relative abundance. 
      \item \textbf{Feature selection and sure screening}. For the microbe-microbe network, we use two 
      methods, SparCC and graphical lasso to reduce the number of edges between pairs of microbes.
      For the microbe-disease network, we use logistic lasso regression to reduce the number of 
      features that are not helpful in predicting disease. 
      \item \textbf{Causal discovery algorithms}. For the microbe-microbe network, we implement
      PC-stable with a max depth of 2. For the microbe-outcome network, we implement CD-NOD
      where the covariates correspond to the heterogeneity index. 
      \item \textbf{Variational autoencoder}. xxx. 
    \end{enumerate}

  \end{block}

  \begin{block}{T2D}

    Et rutrum ex euismod vel. Pellentesque ultricies, velit in fermentum
    vestibulum, lectus nisi pretium nibh, sit amet aliquam lectus augue vel
    velit. Suspendisse rhoncus massa porttitor augue feugiat molestie. Sed
    molestie ut orci nec malesuada. Sed ultricies feugiat est fringilla
    posuere.

    \begin{figure}
      \centering
      \begin{tikzpicture}
        \begin{axis}[
            scale only axis,
            no markers,
            domain=0:2*pi,
            samples=100,
            axis lines=center,
            axis line style={-},
            ticks=none]
          \addplot[red] {sin(deg(x))};
          \addplot[blue] {cos(deg(x))};
        \end{axis}
      \end{tikzpicture}
      \caption{Another figure caption.}
    \end{figure}

  \end{block}

  \begin{block}{PCOS}

    Nulla eget sem quam. Ut aliquam volutpat nisi vestibulum convallis. Nunc a
    lectus et eros facilisis hendrerit eu non urna. Interdum et malesuada fames
    ac ante \textit{ipsum primis} in faucibus. Etiam sit amet velit eget sem
    euismod tristique. Praesent enim erat, porta vel mattis sed, pharetra sed
    ipsum. Morbi commodo condimentum massa, \textit{tempus venenatis} massa
    hendrerit quis. Maecenas sed porta est. Praesent mollis interdum lectus,
    sit amet sollicitudin risus tincidunt non.

    Etiam sit amet tempus lorem, aliquet condimentum velit. Donec et nibh
    consequat, sagittis ex eget, dictum orci. Etiam quis semper ante. Ut eu
    mauris purus. Proin nec consectetur ligula. Mauris pretium molestie
    ullamcorper. Integer nisi neque, aliquet et odio non, sagittis porta justo.

    \begin{itemize}
      \item \textbf{Sed consequat} id ante vel efficitur. Praesent congue massa
        sed est scelerisque, elementum mollis augue iaculis.
        \begin{itemize}
          \item In sed est finibus, vulputate
            nunc gravida, pulvinar lorem. In maximus nunc dolor, sed auctor eros
            porttitor quis.
          \item Fusce ornare dignissim nisi. Nam sit amet risus vel lacus
            tempor tincidunt eu a arcu.
          \item Donec rhoncus vestibulum erat, quis aliquam leo
            gravida egestas.
        \end{itemize}
      \item \textbf{Sed luctus, elit sit amet} dictum maximus, diam dolor
        faucibus purus, sed lobortis justo erat id turpis.
      \item \textbf{Pellentesque facilisis dolor in leo} bibendum congue.
        Maecenas congue finibus justo, vitae eleifend urna facilisis at.
    \end{itemize}

  \end{block}

\end{column}

\separatorcolumn

\begin{column}{\colwidth}

  \begin{exampleblock}{Results}

    A different kind of highlighted block.

    $$
    \int_{-\infty}^{\infty} e^{-x^2}\,dx = \sqrt{\pi}
    $$

    Interdum et malesuada fames $\{1, 4, 9, \ldots\}$ ac ante ipsum primis in
    faucibus. Cras eleifend dolor eu nulla suscipit suscipit. Sed lobortis non
    felis id vulputate.

    \heading{A heading inside a block}

    Praesent consectetur mi $x^2 + y^2$ metus, nec vestibulum justo viverra
    nec. Proin eget nulla pretium, egestas magna aliquam, mollis neque. Vivamus
    dictum $\mathbf{u}^\intercal\mathbf{v}$ sagittis odio, vel porta erat
    congue sed. Maecenas ut dolor quis arcu auctor porttitor.

    \heading{Another heading inside a block}

    Sed augue erat, scelerisque a purus ultricies, placerat porttitor neque.
    Donec $P(y \mid x)$ fermentum consectetur $\nabla_x P(y \mid x)$ sapien
    sagittis egestas. Duis eget leo euismod nunc viverra imperdiet nec id
    justo.

  \end{exampleblock}

  \begin{block}{Conclusion/Future Work}

    Class aptent taciti sociosqu ad litora torquent per conubia nostra, per
    inceptos himenaeos. Phasellus libero enim, gravida sed erat sit amet,
    scelerisque congue diam. Fusce dapibus dui ut augue pulvinar iaculis.

    \begin{table}
      \centering
      \begin{tabular}{l r r c}
        \toprule
        \textbf{First column} & \textbf{Second column} & \textbf{Third column} & \textbf{Fourth} \\
        \midrule
        Foo & 13.37 & 384,394 & $\alpha$ \\
        Bar & 2.17 & 1,392 & $\beta$ \\
        Baz & 3.14 & 83,742 & $\delta$ \\
        Qux & 7.59 & 974 & $\gamma$ \\
        \bottomrule
      \end{tabular}
      \caption{A table caption.}
    \end{table}

    Donec quis posuere ligula. Nunc feugiat elit a mi malesuada consequat. Sed
    imperdiet augue ac nibh aliquet tristique. Aenean eu tortor vulputate,
    eleifend lorem in, dictum urna. Proin auctor ante in augue tincidunt
    tempor. Proin pellentesque vulputate odio, ac gravida nulla posuere
    efficitur. Aenean at velit vel dolor blandit molestie. Mauris laoreet
    commodo quam, non luctus nibh ullamcorper in. Class aptent taciti sociosqu
    ad litora torquent per conubia nostra, per inceptos himenaeos.

    Nulla varius finibus volutpat. Mauris molestie lorem tincidunt, iaculis
    libero at, gravida ante. Phasellus at felis eu neque suscipit suscipit.
    Integer ullamcorper, dui nec pretium ornare, urna dolor consequat libero,
    in feugiat elit lorem euismod lacus. Pellentesque sit amet dolor mollis,
    auctor urna non, tempus sem.

  \end{block}

  \begin{block}{References}

    \nocite{*}
    \footnotesize{\bibliographystyle{plain}\bibliography{poster}}

  \end{block}

\end{column}

\separatorcolumn
\end{columns}
\end{frame}

\end{document}